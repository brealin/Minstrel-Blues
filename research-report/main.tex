%\documentclass{letterpaper,11pt,twoside,twocolumn}
% \documentclass[10pt, conference, letterpaper]{IEEEtran}
% use sig alternate template
\documentclass{sig-alternate}

% \setlength{\oddsidemargin}{-0.3cm}
% \setlength{\evensidemargin}{-0.3cm}
% \setlength{\textwidth}{16cm}
% \setlength{\topmargin}{-0.8cm}
% \setlength{\textheight}{23.0cm}

%\usepackage[margin=.70in]{geometry}

%\pdfpagewidth=8.5in
%\pdfpageheight=11in
%\setlength\paperheight {11in}
%\setlength\paperwidth {8.5in}
%\setlength{\textwidth}{7in}
%\setlength{\textheight}{9.25in}
%\setlength{\oddsidemargin}{-.25in}
%\setlength{\evensidemargin}{-.25in}


%include our custom files
% useful shortcuts
\newcommand{\cref}[1]{Chapter~\ref{#1}}
\newcommand{\xref}[1]{Section~\ref{#1}}
\newcommand{\fref}[1]{Figure~\ref{#1}}
\newcommand{\tref}[1]{Table~\ref{#1}}
\newcommand{\aref}[1]{Algorithm~\ref{#1}}
\newcommand{\eref}[1]{Equation~\ref{#1}}

\newcommand{\crefs}[2]{Chapters~\ref{#1} and~\ref{#2}}
\newcommand{\xrefs}[2]{Sections~\ref{#1} and~\ref{#2}}
\newcommand{\frefs}[2]{Figures~\ref{#1} and~\ref{#2}}
\newcommand{\trefs}[2]{Tables~\ref{#1} and~\ref{#2}}
\newcommand{\arefs}[2]{Algorithms~\ref{#1} and~\ref{#2}}
\newcommand{\erefs}[2]{Equations~\ref{#1} and~\ref{#2}}

\newcommand{\ptitle}[1]{\medskip\noindent{\bf #1 }}
\newcommand{\pskip}{\medskip\noindent}

\newcommand{\ticks}[1]{``#1''}
\newcommand{\ticksde}[1]{{\glqq}#1{\grqq}\xspace}

\newcommand{\first}{\emph{(i)}~}
\newcommand{\second}{\emph{(ii)}~}
\newcommand{\third}{\emph{(iii)}~}
\newcommand{\fourth}{\emph{(iv)}~}
\newcommand{\fifth}{\emph{(v)}~}

\newcommand{\ie}{\emph{i.e.},\xspace}
\newcommand{\Ie}{I.e.,\xspace}
\newcommand{\eg}{\emph{e.g.},\xspace}
\newcommand{\Eg}{For example,\xspace}
\newcommand{\cf}{\emph{cf.}\xspace}
\newcommand{\Cf}{Cf.\xspace}
\newcommand{\etal}{et~al.\xspace}

\newcommand{\perc}{\,\%\xspace}
\newcommand{\pert}{\,\textperthousand\xspace}

%nice symbols to draw arrow (yes) and -(no)
\usepackage{pifont}
\newcommand{\yes}{\checkmark}
\newcommand{\no}{\textendash}

%annotate TODO stuff by color or remove it in case if submission :)     
\newcommand{\todo}[1]{\textit{\textcolor{blue}{[todo]: #1}}} % Comments
\newcommand{\move}[1]{\textit{\textcolor{green}{[move]: #1}}} % Comments
\newcommand{\del}[1]{\textit{\textcolor{red}{[remove]: #1}}} % Comments
%\newcommand{\todo}[1]{}
%\newcommand{\move}[1]{}
%\newcommand{\del}[1]{}



% for PSTRicks
%\usepackage[pdf]{pstricks}
%\usepackage[off]{auto-pst-pdf}

%graphics included tikz & pfg
\usepackage{graphicx}
\usepackage{epsfig,epic,eepic}
\usepackage{psfrag}
\usepackage[USenglish,german]{babel}
\usepackage{color,pstricks}

% To allow me to use letters with accent
\usepackage[latin1]{inputenc}

%Footnotes
\usepackage{endnotes}

% Fonts
\usepackage[T1]{fontenc}
\usepackage{times}

% Math packages
%\usepackage{amsmath,amssymb,amsfonts,amsthm}
\usepackage{url}
%\usepackage[ruled,noline]{algorithm2e}

% More compact lists
\usepackage{paralist}

% But then adapt spacing
\usepackage{setspace}
\setstretch{1.3}

%for nice tables
\usepackage{multirow}
\usepackage{ctable}

%for nice item lists
\usepackage{paralist}

%usepackage comment for block comments
\usepackage{comment}

%sub-picture numbering
\usepackage{subfigure}
\usepackage[right]{eurosym}

%pseudocode package
\usepackage{pseudocode}



% MAIN DOC
\begin{document}

	\title{Minstrel Blues}
	\subtitle{Distributed Joint Power and Rate Control for todays WiFi Networks}

	\numberofauthors{4}
	\author{
		\alignauthor
		Thomas H{\"u}hn
			\titlenote{\texttt{https://github.com/thuehn/Minstrel-Blues}}\\
	    		\affaddr{Technical University of Berlin}\\
			\affaddr{Berlin, Germany}\\
			\email{thomas@inet.tu-berlin.de}
		\alignauthor
		Felix Fitkau\\
	    		\affaddr{LEDE Project}\\
			\affaddr{Berlin, Germany}\\
			\email{nbd@nbd.name}
		\alignauthor
		Denis R{\"o}per\\
	    		\affaddr{LEDE Project}\\
			\affaddr{Berlin, Germany}\\
			\email{denis.roeper@posteo.de}
		\and
		\alignauthor
		Holger T{\"a}ubig\\
	    		\affaddr{University of Bremen}\\
			\affaddr{Bremen, Germany}\\
			\email{holger.taeubig@dfki.de}
	}

	\maketitle
	\sloppy

	\begin{abstract}
	While there has been extensive theoretical work on sophisticated
	joint resource allocation algorithms for wireless networks,
	their applicability to WiFi (IEEE 802.11) networks is very limited.
	With \textit{Minstrel-Blues} we designed, implemented and analyzed
	our decentralized joint rate and power resource allocation algorithm
	within the Linux kernel	running on off-the-shelf WiFi hardware.
	\end{abstract}

	\keywords{Power control, rate control, WiFi, joint ressource allocation}

	%%%
	\section{Introduction}
	With increasing penetration of WiFi in residential areas chaotic
	unplanned deployments are becoming the norm where access points and
	stations operate on the same or nearby channels either due to lack
	of coordination or insufficient	available frequencies.
	%
	Given the rate of deployment, wireless interference is becoming one key
	component for performance degradation due to the broadcast nature of
	wireless communication and limited unlicensed ISM bandwidth.
	%
	The major reason behind such inefficiency is the lack of practical
	resource allocation algorithms that adapt well to the current conditions
	in a wireless network dynamically and select the appropriate
	transmission parameters such as transmission rates and 	power levels.
	%
	Most current practical schemes are rather simplistic and only change a
	single transmission parameter.
	%
	For instance, Transmit Power Control (TPC) works at the WiFi PHY layer
	and commonly assigns a static and rather high power level to all packets.
	%
	A per-link or packet scheme is expected to provide better performance,
	but typically increases complexity and requires	higher-layer information,
	such as medium access state from the Medium Access Control (MAC) layer.
	%
	Therefore, although performance improvements have been shown in theory,
	these ideas are largely uninvestigated in practice.
	%
	In this paper, our main goal is to analyse the performance impact of a
	new joint and per-link rate and power controller in a real WiFi system.
	%
	We designed and implemented a distributed power and rate control
	algorithm, \textit{Minstrel-Blues}, which does not rely on signal
	strength or channel state information, but uses local statistics from
	periodic sampling of different rate and power combinations.  
	%
	Essentially, Minstrel-Blues can run on any WiFi hardware that supports
	packet-level power and rate control capabilities.
	%
	Minstrel Blues decides the data-rate, and consequently, the minimum
	power-level to support the chosen rate using a two-attribute utility
	function based on the throughput and power consumption of all rates.
	%
	To expose the trade-off between throughput and network interference,
	we also introduced a weight parameter for the utility function, which
	tunes the importance of throughput in utility decisions.
	%
	Our results show that if the goal is on maximizing the per-link
	throughput, Minstrel-Blues can significantly reduce transmission power
	necessary to communicate per link, while maintaining the same throughput
	achieved with maximum transmit power.

	%%%
	\section{Related Work}
	\label{s:related}
	Most practical resource allocation algorithms realized on commodity
	WiFi hardware focus on rate control. Most carrier-sense control solutions
	are vendor-specific as there is no agreed-upon standard and power
	control is often considered infeasible~\cite{abdesslem_feasibility_2006,
	shrivastava_feasibility_2007,shrivastava_understanding_2007}.
	%
	In our work on Minstrel-Piano~\cite{Huehn_ICCCN_2012} we designed and
	implemented a practical tpc algorithm on top of Minstrel rate control
	with the goal to reduce the transmission power level as long as the
	throughput is not effected.

	%%%
	\section{Practical Joint Rate and Power Control}
	\label{s:practical-design} 

		%%
		\subsection{System Design Considerations}
		\label{s:system-design}

			\subsubsection{Freedom and Boundaries}
			\label{s:freedom-boundaries}

			\subsubsection{Per-Data-Packet TPC}
			\label{s:tpc-data}

			\subsubsection{MAC-Layer ACK-Packet TPC}
			\label{s:tpc-ack}

		%%
		\subsection{TPC with Linux WiFi Systems}
		\label{s:todays-wifi-systems}


	%%%
	\section{The Algorithm: Minstrel-Blues}
	\label{s:minstrel-blues}

		%%
		\subsection{Minstrel Rate Control}
		\label{s:minstrel-rate}

		%%
		\subsection{Blues Power Control}
		\label{s:minstrel-blues}

		%%
		\subsection{Joint Rate and Power Control}
		\label{s:joint-control}

			%
			\subsubsection{New Utility Function}
			\label{s:utility-function}

		%%
		\subsection{Technical Settings}
		\label{s:technical-settings}

		%%
		\subsection{Open Issues}
		\label{s:open-issues}

	%%%
	%\section{Simulation Results}
	%\label{s:simulations}

	%\subsection{Simulation Set-up}

	%\subsection{Validation}

	%\subsection{Performance Evaluation}


	%%%
	\section{Performance Analysis of Minstrel-Blues}
	\label{s:performance-analysis}

		%%
		\subsection{Measurement System}
		\label{s:measurement-system}

		%%
		\subsection{Validation}
		\label{s:validation}

			\subsubsection{Experimental Setup}
			
			\subsubsection{Results}

			\subsubsection{Discussion}

		%%
		\subsection{Performance Evaluation: Home-Setup}
		\label{s:evaluation}

			\subsubsection{Experimental Setup}

			\subsubsection{Results}

			\subsubsection{Discussion}

		%%
		\subsection{Performance Evaluation: Mesh-Setup}
		\label{s:evaluation}

			\subsubsection{Experimental Setup}

			\subsubsection{Results}

			\subsubsection{Discussion}

		%%
		\subsection{Summary}

	%%%
	\section{Conclusion}
	\label{s:conclusion}



       % Bibliography
	{
	\bibliographystyle{abbrv}
	\small
	\bibliography{my_references}
	}

	%\appendix
	%\include{appendix}

\end{document}
